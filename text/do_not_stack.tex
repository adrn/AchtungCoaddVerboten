\documentclass[12pt,preprint]{aastex}

\newcounter{address}
\setcounter{address}{1}

\newcommand{\project}[1]{\textsl{#1}}
\newcommand{\documentname}{\textsl{Note}}

\title{Pan-STARRS: Don't coadd your data!}
\author{
  Adrian~M.~Price-Whelan\altaffilmark{\ref{col}},
  David~W.~Hogg\altaffilmark{\ref{CCPP},\ref{email}}
}

\altaffiltext{\theaddress}{\stepcounter{address}\label{col} Department of Astronomy, Columbia University, 550 W 120th St., New York, NY 10027, USA}
\altaffiltext{\theaddress}{\stepcounter{address}\label{CCPP} Center
for Cosmology and Particle Physics, Department of Physics, New York
University, 4 Washington Place, New York, NY 10003}
\altaffiltext{\theaddress}{\stepcounter{address}\label{email} To whom
correspondence should be addressed: \texttt{david.hogg@nyu.edu}}

\begin{abstract}
TODO!
\end{abstract}

% From http://arxiv.org/pdf/0807.2515v1.pdf: By applying a
% homogenization process that brings all input images to a common PSF
% before the coaddition, one produces coadd images where the catalog
% depth and completeness information is much easier to extract. We
% have developed tools to carry out PSF homogenization to the median
% seeing on the input single epoch images during the process of
% coaddition to explore whether the completeness and depth are then
% simply related to the coadd image pixel variances.

% An oldie-but-goodie: 
% http://articles.adsabs.harvard.edu/cgi-bin/nph-iarticle_query?1994AJ....107..802F&amp;data_type=PDF_HIGH&amp;whole_paper=YES&amp;type=PRINTER&amp;filetype=.pdf

% From http://arxiv.org/pdf/1011.4059v1.pdf: 
% The approach outlined in Kaiser (2004) is to Fourier
% transform each image and estimate the uth Fourier coefficient of g by
% a weighted average of the uth Fourier coefficient of each image. The
% weighting is accomplished so that the images with better seeing are
% weighted more heavily in the average.

\begin{document}

\section{Introduction}
The co-adding, averaging, or stacking of data has been immensely
successful in astronomy.  Some of the most important results have come
from analyses of stacked images.  For example, the Hubble Deep Field
(citation) and Ultra-deep Field (citation) produced their immensely
deep catalogs from stacked images.  For another, the 2MASS survey
images and catalogs are a product of image averaging (citation).  It
would take volumes to cite all of the important examples.  There are
also many examples of important results of stacking of spectra or
images of objects that are not expected to be identical, but rather
similar in some relevant way; features can be detected frequently in
the stack that could never be detected in any individual contributor
(Hogg specific examples and citations).

Imagine that you have many images of a very faint source; imagine that
this source is so faint that it cannot be detected at reasonable
significance in any of the individual images.  How should you detect,
photometer, and measure the properties of this source?  The temptation
will be to co-add or stack the images, and detect, photometer, and
measure the source in the stacked image.  To be extremely specific,
the temptation is to take at each pixel location $i$ all the images
$j$, each of which has an intensity estimate $I_{ij}$ for that pixel
location and an estimate of the variance $\sigma_{ij}^2$ of the
intensity estimate, and produce the stack estimate $I_i$ for that
pixel by
\begin{equation}
I_i \leftarrow \left[\sum_j \frac{1}{\sigma_{ij}^2}\right]^{-1}
  \,\sum_j \frac{I_{ij}}{\sigma_{ij}^2}
\quad ,
\end{equation}
that is, the inverse-variance-weighted average at pixel location $i$.
This is the best (lowes variance) possible estimate of the intensity.
In what follows, we will show that this approach is lower in precision
than doing the Right Thing (tm), which is to build a model of the
source in all of the individual images.

In ground-based astronomy, each of the images in your set will, in
general, have different point-spread function, different throughput
(transparency), different sky level, different sky and source noise,
and different locations of cosmic ray hits and defects.  Furthermore,
the images will not in general be precisely aligned astrometrically,
both because the images will be---should be---taken at different
offsets and orientations, but also because there are atmospheric
distortions that make even aligned telescope pointings produce
mis-aligned images at the pixel level.  In general, of course, the
images of interest might come from different telescopes, with
different cameras, and be taken in different bands; in these cases the
heterogeneity is even larger.  Unless we can show that analysis of
this set of heterogeneous images is simple, stacking is very tempting.

Going against the temptation to stack the images is the technical
point that stacking clearly throws away information: There is much
more information in the full list of pixels $I_{ij}$ than in the (much
smaller set of) stack pixels $I_i$.  That said, stacking \emph{can} be
justified in an extremely unrealistic limit: If \textsl{(a)}~the
point-spread function never varies, \textsl{(b)}~the source fluxes and
positions and morphologies don't change at all from image to image,
\textsl{(c)}~all the images are properly sky-subtracted and calibrated
(in intensity), \textsl{(d)}~the noise in the image is either Gaussian
or Poisson, and the noise variance is known accurately in every pixel
of every image, \textsl{(e)}~the images have perfectly understood
world coordinate systems and have been interpolated onto a common
pixel grid precisely, and \textsl{(f)}~there are no unmasked
(zero-inverse-variance) defects, cosmic rays, or outliers in the full
list of pixel values $I_{ij}$.

The time-variability point is of great importance in the coming years,
with time-domain astrophysics driving many new large projects.  It is
impossible, in a na\"ive image-stacking context, to measure the
variability or movement of a source that is detectable in the stacked
image but not individually detectable in individual images or shorter
(fewer-image) stacks.  And yet, in many cases, there \emph{will} be
sufficient information in the original pixels to make the measurement.
We demonstrated this for the measurement of faint source proper
motions in repeat imaging from the \project{Sloan Digital Sky Survey}
(Lang et al, 2009): By fitting the individual image pixels, we were
able to measure the proper motions of sources too faint to detect in
any individual image.  Clearly this project could only be achieved
through modeling all of the pixels in all of the exposures.  Image
stacking makes the objective impossible.  What might not be so
obvious---and what we demonstrate in this \documentname---is that even
simple measurements of flux and position can be corrupted by image
stacking, even in well-behaved and properly calibrated and understood
data.

From an engineering perspective, stacking \emph{can} be the right
thing to do: For some experiments, the benefit---in terms of savings
of time and effort---achieved by stacking the images (relative to
doing the Right Thing) can outweigh the costs---in precision and
accuracy.  Consider this \documentname\ a first step at quantifying
the benefits and costs of doing this violence to your hard-earned
data.  The results are so simple, we expect that in practice this
cost--benefit analysis will rarely come down on the side of stacking.

\section{Method}
We begin by simulating a stack of multi-epoch imaging with ideal conditions: perfectly aligned (registered) images of a single, faint point source (star). The star flux is fixed at a value, but the FWHM (e.g. the seeing) of the star in a given image is drawn from a uniform distribution between 1-2 pixels. The star's position is also stationary and, for convenience, set to be the exact middle of the image. Between epochs, the sky brightness is drawn from a uniform distribution and the thermal variance (e.g. read noise) is also drawn from a uniform distribution between 1-1.5 flux units. % APW: flux units??

Once we generate a set of observations, we cumulatively sum the images with different ordering and weighting of the stack. 

\section{Results and discussion}

\clearpage
\begin{table}
~ \hfill \begin{tabular}{rcc|c}
 & \multicolumn{2}{c}{\textbf{stack~options}} & \textbf{fit~options} \\
\hline
 \textbf{sorting} & PSF & $S/N$ & $S/N$ \\
 \small\em (line thickness) & \small\em (2) & \small\em (1) & \small\em (4) \\
\hline
 \textbf{weighting} & 1 & $[S/N]^2$ & --- \\
 \small\em (line color) & \small\em (\#000) & \small\em (\#777) & \small\em (\#000) \\
\hline
 \textbf{smoothing} & on & off & --- \\
 \small\em (line style) & \small\em (solid) & \small\em (dotted) & \small\em (solid) \\
\hline
\end{tabular} \hfill ~
\caption{The $2\times 2\times 2$ options for stacking and the
  line-style options for the plots thereof.\label{tab:legend}}
\end{table}

\end{document}
